\section[Search components of the NSGA-II algorithm]{\label{identificadorReferenciaCruzada}
Search components of the NSGA-II algorithm}

\ \ In addition to the common concepts of single-objective metaheuristics, a multi-objective metaheuristics has to deal with three additional main search problems. Here these components are identified and in the following sections a detailed explanation will be given of the methods NSGA-II uses to address each of the problems. The first problem is the fitness assignment which has the role of guiding the search algorithm towards Pareto optimal solutions for a better convergence. This procedure assigns a scalar-valued fitness to a vector objective function [5]. NSGA-II employs a dominance based approach for this component of the search algorithm. The second search problem is the preservation of diversity where emphasis is on generating a diverse set of Pareto solutions in the objective space and helping the exploration of the fitness space [2]. The NSGA-II here employs a nearest-neighbor approach through calculating the crowding distance. Lastly, elitism guides the search towards preservation and use of elite solutions which allows a robust, fast, and monotonically improving performance of tje metaheuristic.