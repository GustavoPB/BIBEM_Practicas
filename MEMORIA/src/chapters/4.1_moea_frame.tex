\section[MOEA Framework]{\label{identificadorReferenciaCruzada}
MOEA Framework}

\ \ The problem was implemented in the Java programming language. The study used the MOEA Framework free and open source Java library, version 2.1, available from \underline{http://www.moeaframework.org/}. The MOEA Framework contains a comprehensive suite of tools for analyzing the performance of algorithms. It supports both run-time dynamics and end-of-run-analysis. Run-time dynamics capture the behavior of an algorithm throughout the duration of a run, recording how its solution quality and other elements change. End-of-run analysis, on the other hand, focuses on the result of a complete run and comparing the relative performance of various algorithms.

An adjustment had to be made to the existing library to weaken the constraints on the Permutation subclass of the Variable class to permit the crossover operator that was explained in the operators section. Three additional classes were created, one to specify the problem and two classes for the modi ed crossover and mutation operators.
