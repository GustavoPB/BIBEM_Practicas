%
% Frontmatter - Introducción. Los miembros del tribunal que juzgan los PFC's tienen muchas más memorias que leer, por lo que
%	agradecerán cualquier detalle que permita facilitarles la vida. En este sentido, realizar una pequeña introducción,
%	comentar la organización y estructura de la memoria y resumir brevemente cada capítulo puede ser una buena práctica
%	que permita al lector centrarse fácilmente en la parte que más le interesa.
%

\chapter[Introduction]{
	Introduction}
	
\ \ The magic square is a square matrix of size $nXn$ with integer elements between 1 and $n^2$ where the sums of the elements of all rows, columns and the main diagonal are equal. There are $(n2)!$ ways to fill the square. The magic square belongs to the set of combinatorial optimization problems that can be formulated as constraint satisfaction problems (CSPs) [1]. In this practical assignment we have studied whether the magic square problem of size 4�4 can be optimized effectively with the Non-dominated Sorting Genetic Algorithm II (NSGA-II) genetic algorithm.

Genetic algorithms (GAs) belong to a family of stochastic search methods covered by the generic term Evolutionary Computation (EC). The main characteristic of EC techniques is the computational simulation of the natural evolutionary process, as modeled by the Neo-Darwinian paradigm. These stochastic approaches were developed as an alternative to solving high-dimensional, discontinuous and multimodal problems, where traditional deterministic search techniques often proved ineffective [2].

NSGA (Non-Dominated Sorting in Genetic Algorithms) [3] is a popular non-domination based genetic algorithm for multi-objective optimization. It has been proved as very effective but still received criticism for its computational complexity, lack of elitism, and its diversity preservation techniques. A modified version of this algorithm based on its original design is the NSGA-II [4] algorithm. It incorporates elitism, has a better sorting algorithm and less bias in preserving diversity.
